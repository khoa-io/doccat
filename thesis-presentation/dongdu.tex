% Tác giả: Hoàng Văn Khoa
% Email: hoangvankhoa@outlook.com

\section{Định hướng giải pháp}

\subsection{Tách từ Tiếng Việt}

\begin{frame}
  \frametitle{Tách từ Tiếng Việt}
  
  \begin{itemize}[<+->]
    \item phương pháp ghép cực đại: sử dụng từ điển, đặt các từ vào câu sao cho phủ hết câu đó
    \note{ưu điểm: nhanh; nhược điểm: độ chính xác thấp, không xử lí được từ có trong từ điển}
    \item các phương pháp học máy như đồ thị hóa, hidden Markov, conditional random field, maximum entropy
      \begin{itemize}[<+->]
        \item phương pháp Support Vector Machine (SVM) đi kèm pointwise => thư viện Đông Du (C/C++)
        \note{Sử dụng bản sửa đổi của thư viện Đông Du để tiến hành tách từ trên Python}
        \item So sánh dongdu và vnTokenizer
          \begin{flushleft}
            \begin{table}[p]
              \begin{tabular}{|l|l|l|}
                \hline
                Tiêu chí & vnTokenizer & dongdu \\ \hline
                Độ chính xác & 97,2\% & 98,2\% \\ \hline
                Thời gian (giây) & 194,672 & 26,2 \\ \hline
                RAM (MB) & 19,8 & 15,1 \\ \hline
              \end{tabular}
              \caption{So sánh dongdu và vnTokenizer}
            \end{table}
          \end{flushleft}
      \end{itemize}
  \end{itemize}
  
\end{frame}


\subsection{Trích xuất từ khóa}

\begin{frame}
  \frametitle{Trích xuất từ khóa}
  
  \begin{itemize}[<+->]
    \item Xếp hạng từ có ý nghĩa cao hay thấp dựa vào trọng số tf.idf
    \item Từ có ý nghĩa cao là:
    \begin{itemize}
      \item Xuất hiện nhiều lần trong một (lớp) văn bản (đồng biến với số lần sử dụng trong (lớp) văn bản)
      \item Xuất hiện ít trong các (lớp) văn bản khác (nghịch biến với số (lớp) văn bản sử dụng nó)
    \end{itemize}
    \item Công thức tính: $w_{tf.idf}(t, d) = w_{tf}(t, d) * idf(t)$
    \begin{itemize}
      \item \[ w_{t, d} = \left\{
        \begin{array}{l l}
          1 + \log{tf_{t,d}} & \quad \text{nếu $tf_{t,d} > 0$}\\
          0 & \quad \text{nếu ngược lại} 
        \end{array} \right.\]
      \item $tf_{t,d}$ là số lần từ $t$ được sử dụng trong văn bản $d$
      \item $idf(t) = \log{N/df_{t}}$ với $N$ là số văn bản trong bộ dữ liệu, $df_{t}$ là số văn bản chứa từ $t$
    \end{itemize}
  \end{itemize}
  
\end{frame}

\subsection{Giải thuật phân loại Na\"\i ve Bayes}

\begin{frame}
  \frametitle{Tập học và tập kiểm thử}
  
  \begin{itemize}[<+->]
    \item Kích thước tập học
    \item Độ phong phú của tập học
    \item Số từ khóa trích xuất ra
    \item Đánh giá độ chính xác bằng phương pháp hold-out
  \end{itemize}
  
\end{frame}

\begin{frame}
  \frametitle{Giai đoạn phân loại}
  
  \begin{itemize}[<+->]
    \item Giá trị likelihood của văn bản $d$ đối với $c_{i}$: $\log{P(c_{i})} + \log{(\Pi_{t_{j} \in T\_d} (t_j | c_j))} $
    \item Phân lớp văn bản $d$ thuộc lớp $c^{*}$: $c^{*} = argmax_{c_i \in C} (\log{P(c_{i})} + \log{(\Pi_{t_{j} \in T\_d} (t_j | c_j))})$
  \end{itemize}
  
\end{frame}
